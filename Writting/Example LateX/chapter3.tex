%%--------------------Chapter 3------------------------
\chapter{Materials and Methods}
\section{A section}
%%--------------------example citation-----------------
Start writing from here~\cite{ConcreteMath}
\subsection{A subsection}
some text with footnote\footnote{A first}. more
text\footnote{second} and more.

%%-----------------text superscript, subscript--------------------
A sentence with superscript.\textsuperscript{superscript}. A
sentence with subscript.$_{\mbox{\footnotesize{subscript}}}$

%%-----------------printing verbatim--------------------------------
\subsection{Printing Verbatim}
\begin{verbatim}
THIS TEXT WILL BE DIRECTLY PRINTED AS IF TYPED ON A TYPEWRITER, WITH
ALL LINE BREAKS AND SPACES,              without any LaTex command
being executed.

public class BasicsDemo{
    public static void main(String[] args){
        int sum = 0;
        for (int current = 1; current <= 10; current++){
            sum += current;
        }
        System.out.println("Sum = " + sum);
    }
}
\end{verbatim}

\begin{Verbatim}[frame=single, xrightmargin=2pc]
public class BasicsDemo{
    public static void main(String[] args){
        int sum = 0;
        for (int current = 1; current <= 10; current++){
            sum += current;
        }
        System.out.println("Sum = " + sum);
    }
}
\end{Verbatim}

%%---------------program code-------------------------------------
\section{A section}
\subsection{Pretty-printing program code}
\singlespacing
\begin{algorithmic}
\IF {$i\leq0$} \STATE $i\gets1$ \ELSE \IF {$i\geq0$} \STATE
$i\gets0$ \ENDIF \ENDIF
\end{algorithmic}
\subsubsection{Example Java code}
The following is an code example:
\begin{lstlisting}[language=Java]
public class BasicsDemo{
    public static void main(String[] args){
        int sum = 0;
        for (int current = 1; current <= 10; current++){
            sum += current;
        }
        System.out.println("Sum = " + sum);
    }
}
\end{lstlisting}
\doublespacing
